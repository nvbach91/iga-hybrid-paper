
\section{Discussion}
\label{s:discussion}
From the aggregation cube in Figure \ref{fig:cube-aggregation} we can see that there are 8 cases to be considered in our analysis. The goal here is to analyze all cases and find out patterns that could represent the characteristics of a code list, whether the instances are (not) part of a code list, why (not), and whether they could be considered as code list members even though they do not have the defining characteristics.

We have found that if an instance is a \textit{skos:Concept} and part of a \textit{skos:ConceptScheme} then it is by definition a code list member and the \textit{skos:ConceptScheme} is the code list, regardless of whether they have a class or not. There are, however, several cases, e.g., \textit{txn:Status\_Preliminary}, \textit{txn:Status\_Testing}, \textit{txn:TaxonConcept\_Scheme}, \textit{pc:Negotiated}, \textit{pc:Open},  \textit{mil:Cavalry},  \textit{mil:Rank\_Private}, where the instance is a \textit{skos:Concept} but not part of a \textit{skos:ConceptScheme}. These are probably desolate instances that do not belong to a higher categorization but still count as codes, and the last reason is simply the link to \textit{skos:ConceptScheme} is unintentionally missing.

Another observation is regarding the fact that there are instances that have a \textit{skos:ConceptScheme} as their class. These of course are the code lists themselves, but not codes of a code list. 

There are also cases where the \textit{skos:ConceptScheme} and the class have similar naming, e.g., \textit{lawd:Reading} and \textit{lawd:readingScheme}. This could mean that the class being instantiated by the codes may theoretically have the same function as the \textit{skos:ConceptScheme} semantically.

The next case is where the instance has a class but is not modeled entirely using SKOS. There are four types of modeling decisions happening here:
\begin{enumerate}
    \item the class implicitly defines a code list as if it were a \textit{skos:ConceptScheme} while the instances are modeled as \textit{skos:Concept}, examples of this case can be found in the Military ontology, e.g. the class \textit{mil:MilitaryRank} and its instances like \textit{mil:Rank\_Private}, \textit{mil:Rank\_Polkovnik}
    \item the class implicitly (and semantically) defines a code list with its instances enumerating obvious codes without the link to \textit{skos:Concept}, e.g., the classes \textit{geop:geographical\_region}, \textit{keys:Key}, \textit{te:TemporalUnit}, \textit{gr:DayOfWeek}, \textit{rlog:Level}, \textit{pso:PublicationStatus},
    \item the instance is a metadata instance used for an annotation property, in this case, it is highly probable that the class comes from a foreign namespace, e.g. \textit{foaf:Person} for contact info,
    \item the instance is an exemplary instance of a class for showcasing purposes, e.g., \textit{ostop:scotland}, in which case this class should be from the same namespace as the instance.
\end{enumerate}

The last case is when an instance does not have a class nor is modeled as SKOS. These are instances of one of the classes in Code listing \ref{lst:sparql13}, which are not relevant in finding code lists, and therefore are ignored during the querying process.

The multi-dimensional analysis in Section~\ref{s:analysis_code_list_modeling_practice} represented a complementary view on the individuals from LOV vocabularies, with respect to the SPARQL extraction patterns from Section~\ref{s:codelistanalyzer}.
It leads us to also check whether the `negatives' wrt. the queries are true or false negatives, i.e., brought us at least qualitative (and satisfactory) insights related to the recall of these queries: the individuals not covered by the queries do not appear as codes making part of code lists.
%Since the tooling used in the multi-dimensional analysis was not confined to SPARQL, we added the namespace correspondence as a third dimension, which 


%\begin{table}[]
%\footnotesize
%\centering
%\begin{tabular}{|l|l|}
% \hline
%owl:Thing              & skos:ConceptScheme \\ \hline
%owl:Ontology           & voaf:Vocabulary    \\ \hline
%owl:NamedIndividual    & rdf:Property       \\ \hline
%owl:ObjectProperty     & rdfs:Class         \\ \hline
%owl:AnnotationProperty & owl:Class          \\ \hline
%owl:DatatypeProperty   & rdfs:Datatype      \\ \hline
%owl:OntologyProperty   & rdfs:Resource      \\ \hline
%\end{tabular}
%\caption{List of ignored classes}
%\label{tab:ignored classes}
%\end{table}


%1 - skriptem

%2 - multidimenzialne predtim nenapadla (shoda namespace)



