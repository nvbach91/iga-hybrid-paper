%%
%% The abstract is a short summary of the work to be presented in the
%% article.
\begin{abstract}
%The Semantic Web stack is currently rapidly evolving  thanks to the currently eminent and influential advancement of knowledge graphs in combination with ontology schemas, which provide contextual semantics, and code lists, which enhance interoperability and re-use within systems. Among these data-modeling patterns, code lists have the simplest structures found in a knowledge base, but, in many cases, they play a significant role in organizing knowledge in a straightforward way. In this paper, we introduce our motivation for 
While the early phase of the Semantic Web put emphasis on conceptual modeling through ontology classes, and the recent years saw the rise of loosely structured, instance-level knowledge graphs (used even for modeling concepts), in this paper we focus at a third kind of concept modeling: via code lists, primarily those embedded in ontologies 
%exploring the concept of code lists in ontologies 
and vocabularies.
%and we provide an analysis of occurrence, structure, and role of these code lists in contrast to other RDF data-modeling patterns.
%First, we attempt to characterize (and illustrate on examples) the typical ways of concept modeling in these different kinds of knowledge bases.
We attempt to characterize the candidate structures for code lists based on our observations in OWL ontologies.
Our main contribution is then an approach implemented as a series of SPARQL queries and a lightweight web application that can be used to browse and detect potential code lists in ontologies and vocabularies, in order to extract and enhance them, and to store them in a stand-alone knowledge base. 
%We validate our approach by implementing a web 
The application allows inspecting query results as coming live from the Linked Open Vocabularies catalog. In addition, we describe a complementary bottom-up analysis of potential code lists. We also provide in this paper a demonstration of the dominant nature of embedded codes from the aspect of ontological universals and their alternatives for modeling code lists.

%Code lists play a significant role in areas such as open fiscal data where they provide a simple way to refer to diverse budget concepts and also they can serve as a linkage between terms to facilitate comparative analysis.

\end{abstract}

%%
%% Keywords. The author(s) should pick words that accurately describe
%% the work being presented. Separate the keywords with commas.
\begin{keyword}
code list, knowledge base, knowledge graph, knowledge representation, ontology, RDF, Semantic Web
\end{keyword}