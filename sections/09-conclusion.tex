\section{Conclusions}
\label{s:conclusion}
%The presented approach...
In this paper, we have described our workflow and implementation for the semi-automatic detection, extraction, and analysis of code list structures embedded in OWL ontologies. 
%In this paper, we have showcased our experimental approach and our 
The implementation is primarily based on SPARQL queries.
%for the detection and analysis of code list structures in the chosen knowledge base LOV. 
As input material we chose the collection of ontologies indexed in the LOV catalog.
The tool is freely accessible, and so is also the result of its run on the LOV resource.

%Next, we have discussed the motivation for analyzing each structural part of the RDF knowledge representation including a brief comparison for these structural parts from our point of view.
The design of the queries was informed by a comprehensive analysis of possible patterns that could express the general notion of code list in OWL.
Looking at the semantics of code lists, we identified the phenomenon of individuals frequently representing ontological universals. 
This led us to a contextual, qualitative study of alternative ways the designers can choose to represent concepts (universals).
We consequently focused on code lists represented as sets of syntactic individuals embedded in vocabularies in the LOV database.
Under this optic, we also carried out a complementary bottom-up analysis of all such individuals, quantified their position in a multi-dimensional space of features, and qualitatively analysed the different cases.

%Lastly, we have provided results achieved from our workflow, revealing a comprehensive collection of statistical information involving code list occurrences and their structure in LOV.

%Since the research is a part of a larger effort in studying the interplay of concept modeling in `hybrid' representations, having concepts (universals) as classes, code list individuals and possibly even `knowledge-graph-style' individuals, in a single (though, modular) knowledge base, the introductory part of the paper also provided a broader view of this modeling landscape.

We are aware of limitations of our approach, which we aim to overcome in our future work.
Foremost, while the individuals embedded in ontologies were a relatively easy target, they only represent a fraction of structures potentially usable as stand-alone, SKOS-conformant code lists on the Semantic Web.
We assume that a further wealth of code list structures could be obtained by transformation from classes, and possibly also by extraction from large knowledge graphs. 
This will however require additional patterns to be formulated and tested, and it might be hard to reach satisfactory accuracy.
Furthermore, while we believe that the final result produced by our workflow, namely, a dataset of stand-alone code lists, will find its consumers, we have not yet elaborated techniques by which data and knowledge engineers could be informed about the existence of a code list satisfying their needs. 
The practice of communities that routinely handle code lists (even beyond the Semantic Web), such as librarians or public officers, should be studied in order to offer them a mechanism to retrieve code lists built on the basis of ontologies and knowledge graphs, and this way interconnect them to the Semantic Web effort.

%Our future work will focus on: (1) querying and analyzing additional open knowledge graphs with an available SPARQL endpoints found in SPARQLES \cite{DBLP:journals/semweb/VandenbusscheUM17} or other datasets; (2) the ontological analysis of the meaning of codes, namely how often they correspond to ontological universals or particulars; (3) the lexical analysis of patterns within the embedded code list structure wrt their overarching class and the property they provided values for; (4) and finding the motivations for incorporating code lists into the Semantic Web rather than preserving an ontology-instance knowledge graph dichotomy.

%%
%% The acknowledgments section is defined using the "acks" environment
%% (and NOT an unnumbered section). This ensures the proper
%% identification of the section in the article metadata, and the
%% consistent spelling of the heading.
\section*{Acknowledgement}
This research was supported from the institutional-support fund for long-term conceptual development of science and research at the Faculty of Informatics and Statistics of the Prague University of Economics and Business (IP400040), and by the project IGA V\v{S}E \textnumero\ F4/33/2019.

